\documentclass[10pt,DIV12]{scrartcl}
%\documentclass[a4paper,10pt]{paper}

\usepackage[english,portuges]{babel}
\usepackage[latin1]{inputenc}
\usepackage{fancyhdr}
\usepackage[nofancy]{svninfo}
\usepackage{enumerate}
\usepackage{hyperref}
\usepackage[all]{xy}
\usepackage{ifdraft}

%\usepackage[utf8x]{inputenc}
\usepackage[top=2cm,left=2cm,right=2cm,bottom=2cm]{geometry}
\usepackage{indentfirst}
\usepackage{amsfonts, amsmath, amssymb}
%\usepackage[brazil]{babel}
%\usepackage[T1]{fontenc}%codifica��o
\usepackage{graphicx,color}
\usepackage{comment}
\usepackage{amssymb}
\usepackage{ifthen}

\parindent0pt

\lhead{\small Universidade Federal do Rio Grande do Sul\\
Instituto de Inform�tica\\
Departamento de Inform�tica Te�rica}
\rhead{\small INF05510 -- Otimiza��o Combinat�ria\\
Profa. Luciana S. Buriol\\}


\begin{document}

\newboolean{showComments}
%mostrar comentarios
%\setboolean{showComments}{true}
\setboolean{showComments}{false}


\begin{center}
  \Large
Exemplo de texto editado em latex
\end{center}



\pagestyle{fancy}

\bigskip
Exemplo de um sistema e um dicion�rio:

\begin{align*}
  \textbf{max}\qquad & z = 6x_1+8x_2+5x_3+9x_4\\
  \textbf{sujeito a}\qquad
  & 2x_1+x_2+x_3+3x_4\leq 5\\
  & x_1+3x_2+x_3+2x_4\leq 3\\
  & x_1,x_2,x_3,x_4\geq 0
\end{align*}

\begin{align}
  \textbf{max}\qquad & z = 6x_1+8x_2+5x_3+9x_4 \label{ex:fo}\\
  \textbf{sujeito a}\qquad
  & w_1 = 5 - 2x_1 - x_2 - x_3 - 3x_4 \label{ex:r1}\\
  & w_2 = 3 - x_1  - 3x_2 - x_3 - 2x_4 \label{ex:r2}\\
  & x_1,x_2,x_3,x_4,w_1,w_2\geq 0\notag
\end{align}



\ifthenelse {\boolean{showComments}}{
  \textbf{Resposta:}\\
  Vari�veis:\\
AQUI DENTRO PODEM COLOCAR COMENT�RIOS QUE N�O APARECER�O NO PDF
}



\begin{enumerate}

\item Uma empresa possui dois produtos $P_1$ e $P_2$. O seu departamento de marketing estuda a forma mais econ�mica de aumentar em 30\% a vendas de cada um dos produtos. As alternativas s�o:
\begin{itemize}
   \item Investir em um programa institucional com outras empresas do mesmo ramo. Esse programa deve proporcionar um aumento de 3\% nas vendas de cada produto, para cada \$ 1.000,00 investidos.
   \item Investir diretamente na divulga��o de cada produto:
    \begin{itemize}
     \item Cada \$ 1.000,00 investidos em P1 retornam um aumento de 4\% em sua venda.
     \item Cada \$ 1.000,00 investidos em P2 retornam um aumento de 10\% em sua venda. 
    \end{itemize}
\end{itemize}
A empresa disp�e de \$ 10.000,00 para esse empreendimento. Quanto dever� destinar a cada atividade? Construa o modelo do sistema descrito.

%~~~~~~~~~~~~~~~~~~~~~~~~~~~~~~~~~~~~~~~~~~~~~~~~~~~~~~~~~~~~~~~~~~~~~~~~~~~~~~~~~~~~~~~~~~~~
\item O setor de transporte de cargas da VASP operando em S�o Paulo disp�e de 8 avi�es B-727, 15 avi�es ELECTRA e 12 avi�es BANDEIRANTE para v�os amanh�. H� cargas para remeter amanh� para o Rio de Janeiro (150 ton) e Porto Alegre (100 ton). Os custos operacionais de cada avi�o e suas capacidades (tonelagem) s�o especificados na tabela abaixo:\\

\begin{table}[htb]
   \centering
   \begin{tabular}{|l|c|c|c|}
      \hline
                      & B-727& Electra & Bandeirantes\\
      \hline
      SP $\rightarrow$ Rio        & 23   & 5       & 1,4\\
      \hline
      SP $\rightarrow$  P. Alegre & 58   & 10      & 3,8\\
      \hline
      Tonelagem       & 45   & 7       & 4  \\
      \hline
   \end{tabular}
\end{table}

Quanto e quais avi�es devem ser mandados para o Rio e Porto Alegre para satisfazer a demanda e minimizar os custos?

\end{enumerate}


\end{document}
% Local Variables:
% auto-fill-function: do-auto-fill
% fill-column: 110
% mode-name: "LaTeX"
% TeX-PDF-mode: t
% End:

% LocalWords:  UFRGS Simplex Bland
